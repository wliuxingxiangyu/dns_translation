\documentclass[english,runningheads,a4paper]{llncs}[2018/03/10]

\usepackage[ngerman,english]{babel}
\usepackage[UTF8]{ctex}

\usepackage{graphicx}
\usepackage{upquote}
\usepackage{booktabs}
\usepackage{paralist}
\usepackage{csquotes}
\usepackage{textcmds}
\usepackage{xcolor}
\usepackage{indentfirst}
\usepackage{changepage}


\begin{document}
%------------DNS Security :page24 start-------------------
\par\noindent\textbf{资源记录}

\par\noindent 本章从一般到具体进展,本节是最具体的,它也是DNS的核心。 DNS的主要功能是共享信息,方便和自动解析网站www.example.com或发送邮件至user@example.com。

\par\setlength\parindent{2em}资源记录(RESOURCE RECORD,以下简称“RR”)是域文件中的条目,用于提供用户正在查找的信息。DNS内置的所有可用性都只是因为RR 中包含的信息可以从权威服务器传输到递归服务器,然后人们可以愉快地访问网站、共享文件等。

\par\setlength\parindent{2em}RR是服务于特定功能的域名的域文件中的条目。有许多不同类型的RR,但实际上通常只有大约10种。 我们已经审查过的常见RR之一是SOA。 SOA的独特之处在于它是唯一不遵循标准格式的常见RR。

\par\setlength\parindent{2em}RR的标准格式由五个字段组成,每个字段包含特定的字段关于RR的信息:

\par\setlength\parindent{2em}Name \qquad TTL \qquad Class \qquad Type \qquad Data

\par\setlength\parindent{2em} 在实际中,格式通常如下所示:

\par\setlength\parindent{2em} mail.example.com. \qquad 3600 \qquad IN \qquad A \qquad 10.10.100.102

%------------DNS Security :page25 start-------------------
\par\noindent\rule[0.25\baselineskip]{\textwidth}{1pt} %分割线rule

\begin{adjustwidth}{2em}{2em}
\qquad 这就是尾随点的突出重要性的地方。 如果域之后没有尾随点,则DNS软件通常会将其解释为指示主机名,并且软件将在域上添加。 在上面的示例中,如果mail.example.com之后没有尾随点,则软件会认为这条RR是mail.example.com.example.com。
\end{adjustwidth}

\par\noindent\rule[0.25\baselineskip]{\textwidth}{1pt} %分割线rule

\par\setlength\parindent{2em} Name字段是主机名还是IP地址取决于这条记录;该字段是代表这个特定RR的对象。 TTL字段仅在当前TTL与特定RR所需的SLA中定义的TTL不同的时候才使用。 如果不指定TTL使用默认域的TTL。 Class字段的值始终为“IN(Internet)”,这是DNS目前唯一支持的类别。 Type字段定义了RR的目的,在上面的例子中,“A”记录是一个地址记录,其他Type字段将在下面讨论。Data字段取决于RR的类型; Data字段的值是DNS分享有关该特定RR的信息。

\vbox{}

\par\noindent\textbf{地址记录}

\par\noindent 最常见的RR类型是地址(Address records,以下简称“A记录”)。 A记录将FQDN映射到IP地址,如上例所示。 每个主机地址在一个域中必须具有A记录(域文件中的主机但不属于域的一部分,不需要该域文件中的A记录)或规范名称记录:

\par\setlength\parindent{2em}mail.example.com. \qquad 3600 \qquad IN \qquad A \qquad 10.10.100.102

\par\setlength\parindent{2em}请记住,TTL字段是可选的,如果RR可以使用区域的默认TTL,则无需将其包含在RR中。最基本的RR必须含A,无疑是最常用的。 即使是简单的区域文件通常也会有5或6个A记录。

\par\setlength\parindent{2em}如果FQDN指向多个IP地址,则名称服务器将以循环方式返回它们。请求A记录的第一台服务器将获得第一条记录,第二台服务器将获得第二条记录,第三台服务器将获得第三条记录,等等。一旦名称服务器返回了所有A记录,它将在开始的地方重新开始。

\par\setlength\parindent{2em}还有专门为IPv6地址设计的A记录。此记录称为AAAA记录,其格式与A记录相同,但其具有IPv6地址:

\par\setlength\parindent{2em}ipv6.example.com. \qquad 3600 \qquad IN \qquad AAAA \qquad 2001:468:504:1:210:5aff:fe1a:11e

\vbox{}

\par\noindent\textbf{典型名称记录}

\par\noindent 规范名称记录(Canonical Name,以下简称“CNAME”)是将一个FQDN映射到另一个FQDN的别名。 它不是直接将FQDN映射到IP地址,而是更容易映射到另一个host主机。如果有许多FQDN指向相同的IP地址,当对主地址进行更改时,其他地址会自动更新,这一点尤其有用。 CNAME看起来与此类似:

%------------DNS Security :page26 start-------------------

\end{document}
