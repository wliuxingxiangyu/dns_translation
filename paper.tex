\documentclass[english,runningheads,a4paper]{llncs}[2018/03/10]

\usepackage[ngerman,english]{babel}
\usepackage[UTF8]{ctex}

\usepackage{graphicx}
\usepackage{upquote}
\usepackage{booktabs}
\usepackage{paralist}
\usepackage{csquotes}
\usepackage{textcmds}
\usepackage{xcolor}
\usepackage{indentfirst}
\usepackage{changepage}


\begin{document}




%------------DNS Security :page24 start-------------------
\par\noindent\textbf{资源记录}

\par\noindent 本章从一般到具体进行阐述,而本节是最具体的,也是DNS的核心。DNS 的主要功能是共享信息,方便和自动解析网站www.example.com或发送邮件至user@example.com。

\par\setlength\parindent{2em}资源记录(RESOURCE RECORD,以下简称“RR”)是域文件中的条目,用于提供用户正在查找的信息。DNS内置的所有可用性都只是因为RR中包含的信息可以从权威服务器传送到递归服务器,然后人们可以愉快地访问网站、共享文件等。

\par\setlength\parindent{2em}RR是服务于特定功能的域名的域文件中的条目。有许多不同类型的RR,但实际上通常只有大约10种。我们已经查阅过的常见RR之一是SOA。SOA的独特之处在于它是常见的唯一不遵循标准格式的RR。

\par\setlength\parindent{2em}RR的标准格式由五个字段组成,每个字段包含特定的字段关于RR的信息:

\par\setlength\parindent{2em}Name \qquad TTL \qquad Class \qquad Type \qquad Data

\par\setlength\parindent{2em} 在实际中,格式通常如下所示:

\par\setlength\parindent{2em} mail.example.com. \qquad 3600 \qquad IN \qquad A \qquad 10.10.100.102

\vbox{}

%------------DNS Security :page25 start-------------------
\par\noindent\rule[0.25\baselineskip]{\textwidth}{1pt} %分割线rule中间有文字 ---start---

\begin{adjustwidth}{2em}{2em}
\qquad 这就是尾随点的凸显重要性的地方。如果域名之后没有尾随点,则DNS解析软件通常会将其解析为主机名,并且软件将在域上添加。在上面的示例中,如果mail.example.com 之后没有尾随点,则DNS解析软件会认为这条RR是mail.example.com.example.com。
\end{adjustwidth}

\par\noindent\rule[0.25\baselineskip]{\textwidth}{1pt} %分割线rule---end---

\par\setlength\parindent{2em} Name字段是主机名还是IP地址取决于RR这条记录;该字段代表着一个特定RR的对象。TTL字段仅在当前TTL与特定RR所需的SLA 中定义的TTL不同的时候才使用。如果不指定TTL就使用默认域的TTL。Class字段的值始终为“IN(Internet)”,这是DNS目前唯一支持的类别。Type字段定义了RR的目的,在上面的例子中,“A”记录是一个地址记录,其他Type字段将在下面讨论。Data字段取决于RR的类型;Data字段的值是DNS表示有关该特定RR的信息。

\vbox{}

\par\noindent\textbf{地址记录}

\par\noindent 最常见的RR类型是地址(Address records,以下简称“A记录”)。A记录将FQDN映射到IP地址,如上例所示。每个主机地址在一个域中必须具有A记录(其中域文件中的主机名但不属于该域的一部分时,不需要该域文件中的A记录)或规范名称记录:

\par\setlength\parindent{2em}mail.example.com. \qquad 3600 \qquad IN \qquad A \qquad 10.10.100.102

\par\setlength\parindent{2em}请记住,TTL字段是可选的,如果RR可以使用区域的默认TTL,则无需将其包含在RR中。最基本的RR必须含A,这是最常见的。即使是简单的域文件通常也会有5或6个A记录。

\par\setlength\parindent{2em}如果FQDN指向多个IP地址,则名称服务器将以循环方式返回它们。请求A 记录的第一台服务器将获得第一条记录,第二台服务器将获得第二条记录,第三台服务器将获得第三条记录,以此类推。一旦名称服务器返回了所有A记录,它将在开始的地方重新开始。

\par\setlength\parindent{2em}还有专门为IPv6地址设计的A记录。此记录称为AAAA记录,其格式与A记录相同,但其具有IPv6地址:

\par\setlength\parindent{2em}ipv6.example.com. \quad 3600 \quad IN \quad AAAA \quad 2001:468:504:1:210:5aff:fe1a:11e

\vbox{}

\par\noindent\textbf{典型名称记录}

\par\noindent 规范名称记录(Canonical Name,以下简称“CNAME”)是将一个FQDN映射到另一个FQDN的别名。它不是直接将FQDN映射到IP地址,而是更容易映射到另一个host主机。如果有许多FQDN指向相同的IP 地址,当对主地址进行更改时,其他地址会自动更新,这一点尤其有用。CNAME看起来与此类似:

%------------DNS Security :page26 start-------------------
\par\setlength\parindent{2em}www.example.com. \qquad 3600 \qquad IN \qquad CNAME \qquad mail.example.com.

\par\setlength\parindent{2em}mail.example.com. \qquad 3600 \qquad IN \qquad A \qquad 10.10.100.102

\par\setlength\parindent{2em}为了使CNAME工作,它必须指向具有有效A记录的FQDN。CNAME不必指向同一域文件中的FQDN;它可以指向其他FQDN:

\par\setlength\parindent{2em}www.example.com \qquad 3600 \qquad IN \qquad CNAME \qquad www.foo.com.

\par\setlength\parindent{2em}同样,www.foo.com必须有一个有效的A记录才能使用这个CNAME。关于CNAME记录的另一个重要的注意点是它们不能被其他RR使用。MX和NS记录都不能指向CNAME记录。

\vbox{}

\par\noindent\textbf{邮件交换记录}

\par\noindent 邮件交换器(Mail exchanger records,以下简称“MX记录”)记录用于定义应发送域邮件的主机。与CNAME记录一样,MX记录不必指向域内的FQDN,MX记录可以指向任何主机,域内或者域外,只要该主机设置为接收该域的邮件。MX记录必须指向由A记录表示的FQDN而不是CNAME。

\par\noindent\rule[0.25\baselineskip]{\textwidth}{1pt} %分割线rule中间有文字 ---start---

\begin{adjustwidth}{2em}{2em}
\qquad 大多数现代邮件传输代理(MTA)都了解CNAME记录并可以将邮件转发给它们,但有些仍然没有。它也违反了RFC协议规定,所以请不要这样做。
\end{adjustwidth}

\par\noindent\rule[0.25\baselineskip]{\textwidth}{1pt} %分割线rule---end---

\par\setlength\parindent{2em}MX记录看起来像这样:

\par\setlength\parindent{2em}example.com. \qquad 3600 \qquad IN \qquad MX \qquad 10mail.example.com.

\par\setlength\parindent{2em}example.com. \qquad 3600 \qquad IN \qquad MX \qquad 20mail.example.com.

\par\setlength\parindent{2em}有一个字段是MX记录独有的,称为记录权重。该权重用于确定多个MX记录的首选项。若记录的权重越低,则对该记录的偏好越大。在上面的示例中,mail.example.com优先于mail.foo.com。试图将邮件发送到user@example.com的MTA 将首先尝试mail.example.com,只有在mail.example.com 超时或退回邮件时才尝试mail.foo.com。

\vbox{}

\par\noindent\textbf{名称服务器记录}

\par\noindent 除了注册为主机之外,名称服务器还必须在域文件中定义为名称服务器(NS)记录。
正如其名,NS记录用于列出域的权威名称服务器。NS记录通常是SOA之后的第一个RRS,其格式如下:
%------------DNS Security :page27 start-------------------
\par\setlength\parindent{2em}example.com. \qquad IN \qquad NS \qquad ns1.example.com.

\par\setlength\parindent{2em}example.com. \qquad IN \qquad NS \qquad ns2.foo.com.

\par\setlength\parindent{2em}example.com. \qquad IN \qquad NS \qquad ns2.example.com.

与MX和CNAME记录一样,NS记录不必指向主机文件中存在的FQDN,NS记录可以指向任何具有有关域认证信息的主机。NS记录必须是一个FQDN,不能是一个IP地址;它还必须指向一个FQDN,该FQDN是A记录,而不是CNAME 记录。

NS记录还有另一个用途。对于能够使用的DNS守护进程,主名称服务器会将对域文件的更新推送到(该区文件中列出的)名称服务器。当然,这在很大程度上依赖于现有的DNS基础设施。

\vbox{}

\par\noindent\textbf{指针记录}

\par\noindent 指针(ptr)记录与记录相反。ptr记录将IP地址映射到域名。ptr 记录存储在称为in-addr.arpa域文件。IP地址块的数据信息通过区域网络协调中心(NCC)分发。

\par\setlength\parindent{2em}目前有五个网络中心:美国互联网号码注册处负责处理北美的信息;拉丁美洲和加勒比的互联网地址注册处处理拉丁美洲和Carribean的IP地址;Re'Seaux IP Europe'ens管理欧洲的IP 地址;非洲网络信息中心是非洲大陆的互联网注册处;以及亚太网络信息中心,负责亚太地区的IP地址。

\par\setlength\parindent{2em}有关IP地址分配的信息处理方式与域名信息几乎相同,并且DNS结构相同。不同的是,管理员使用的是IP地址块,而不是域名,并且区域文件在这方面是不同的。in-addr.arpa 区域文件是使用IP块命名的,后面跟反向按地址ARPA。 如果组织被分配IP块10.100.50.0/24(C类网络块),包含该网络块信息的区域文件将被命名为:

\par\setlength\parindent{2em}50.100.10.IN-ADDR.ARPA

\par\setlength\parindent{2em}区域文件通常只包含三种类型的资源记录:SOA、NS和PTR 记录,其中PTR 记录是主要的关注点。PTR记录采用与转发域文件相同的常规格式,但主机名在数据字段中:

\par\setlength\parindent{2em}102 \qquad 3600 \qquad IN \qquad PTR \qquad mail.example.com.

%------------DNS Security :page28 start-------------------

\par\setlength\parindent{2em}此记录声明IP地址10.10.100.102映射到mail.example.com。与一个主机名可以映射到多个IP地址的记录不同,一个IP地址只能映射到一个主机名。

\par\setlength\parindent{2em}PTR记录信息通常用于身份验证。有些邮件管理员拒绝来自具有反向记录服务器的邮件,有些FTP服务器拒绝来自没有反向记录映射到其主机名用户的登录。这种安全的合理性是有争议的,但它确实存在,而且值得注意。

\vbox{}

\par\noindent\textbf{主机信息记录}

\par\noindent 主机信息(Host Info,hinfo)记录现在不经常使用,但它们偶尔会弹出。 HINFO提供关于主机操作系统和硬件信息。 格式与其他RR相同,但数据字段包含非结构化主机数据:

\par\setlength\parindent{2em}mail.example.com. \quad 3600 \quad IN \quad HINFO \quad “Dell 1650” “Redhat 9.1”

\par\setlength\parindent{2em}RFC 1035建议使用HINFO数据字段的格式,但可以不严格遵循,并且可以替换其他信息。

\vbox{}

\par\noindent\textbf{服务器记录}

\par\noindent 主机信息(Host Info,hinfo)记录现在不经常使用,但它们偶尔会弹出。 HINFO提供关于主机操作系统和硬件信息。 格式与其他RR相同,但数据字段包含非结构化主机数据:

\par\setlength\parindent{2em}不幸的是,并不总是能够知道在一个域名下的主机能支持哪些服务。发布SRV不是为了进行随意的猜测,而是帮助请求者在不知道服务器信息的情况下获得所需的服务。SRV RR的格式如下:

\par\setlength\parindent{2em}Service.Proto.Name \ TTL \ Class \ SRV \ Priority \ Weight \ Port \ Target

\par\setlength\parindent{2em}与MX记录一样,SRV记录允许DNS管理员为各种SRV记录分配不同的权重。 一个真实的例子是使用SRV记录来平衡Web服务器之间的流量:

\par\setlength\parindent{2em}http.tcp.www.example.com. \ IN \ SRV \ 10 10 80 host1.example.com.

\par\setlength\parindent{2em}http.tcp.www.example.com. \ IN \ SRV \ 10 10 80 host2.example.com.

%------------DNS Security :page29 start-------------------

\par\setlength\parindent{2em}在上面的示例中,DNS管理员负载平衡两个服务之间的HTTP服务。当然,必须有host1.example.com和host2.example.com的A记录。 在这种情况下,由于管理员希望在两台服务器之间均匀分布负载,因此权重和优先级设置相同。 如果其中一个服务器需要承担更多的负载,其权重将设置得更低。 同样,如果一台服务器是主服务器,而另一台服务器只是故障转移服务器,则主服务器的优先级将设置为低于故障转移服务器的优先级。

\par\setlength\parindent{2em}SRV记录并不常见; 它们主要用于内部网络服务,例如Microsoft的Active Directory。

\vbox{}

\par\noindent\textbf{文本记录}

\par\noindent 文本(TXT)记录是另一种不常用的RR。 文本记录是用于提供人可读信息的自由格式文本,更加一般是一条条目或一个域,它们通常设置类似于:

\par\setlength\parindent{2em}allan \ IN \ TXT \ ”Hello World!”

\par\setlength\parindent{2em}仍然大量使用TXT记录的一个领域是基于DNS的领域恶意软件检测领域。特别是,恶意软件使用DNS作为其exfil路径,数据和命令通常作为TXT记录嵌入DNS查询参数中。

\vbox{}

\par\noindent\rule[0.25\baselineskip]{\textwidth}{1pt} %“总结”上面的分割线
\par\noindent\textbf{总结}
\par\setlength\parindent{2em}DNS是一个复杂的主题,不能完全涵盖在单个主题的章节中。本章旨在作为DNS及其复杂性的一个较好的概述。在线资源和书中有许多可供DNS及其复杂性的更详细介绍。然而,本章中的信息是一个很好的基线,它包含的信息DNS管理员应该熟悉。

\vbox{}

\par\noindent\rule[0.25\baselineskip]{\textwidth}{2pt} %“注意”上面的分割线
\par\noindent\textbf{注意}
\begin{enumerate}
\item 有些人会告诉你注册的第一个域名是Symbolics.com或者Think.com,你可以告诉他们这些域名是分别注册于1985年的3月和1985年5月,至少在NORDU.NET注册后的3 个月。
\item 当然,这是假设与主机关联的域名也注册了。
\end{enumerate}

\end{document}
